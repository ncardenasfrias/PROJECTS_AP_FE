\documentclass[hidelinks,12pts]{article}
\usepackage[export]{adjustbox}
\usepackage{amsmath}
\usepackage{amssymb}
\usepackage[title,titletoc]{appendix}
\usepackage{array}
\usepackage[english]{babel}
\usepackage{bbm}
\usepackage{blindtext}
\usepackage{bookmark}
\usepackage{booktabs}
\usepackage{cancel}
\usepackage{caption}
\usepackage{csquotes}
\usepackage{enumerate}
\usepackage{enumitem}
\usepackage{epigraph}
\usepackage[left]{eurosym}
\usepackage{float}
\usepackage[bottom]{footmisc}
\usepackage[margin=0.9in]{geometry}
\usepackage{graphicx}
\usepackage{hyperref}
\usepackage{indentfirst}
\usepackage{lscape}
\usepackage{mathtools}
\usepackage{mdframed}
\usepackage{multirow}
\usepackage{multicol}
\usepackage[sort]{natbib}
\usepackage{parskip}
\usepackage{setspace}
\usepackage{subcaption}
\usepackage{titlesec}
\usepackage{tgpagella}
\usepackage{varwidth}
\usepackage{verbatim} %comment block
\usepackage{wrapfig}
\usepackage[dvipsnames]{xcolor}
\usepackage{xltabular}

\renewcommand{\epigraphsize}{\normalsize}
\setlength{\epigraphwidth}{0.7\textwidth}
\renewcommand{\textflush}{flushright}
\renewcommand{\sourceflush}{flushright}


\renewcommand\appendixpagename{Mathematical Appendix}
\renewcommand{\baselinestretch}{1.45}

\hypersetup{
    colorlinks=true, 
    urlcolor= Violet, 
    linkcolor=Black, 
    citecolor=Blue, 
    filecolor = Blue
    } 

%\usepackage{natbib}
\bibliographystyle{plainnat}
%\bibdata{My Library.bib}
%\usepackage[backend=biber, style=authoryear-icomp]{biblatex}

\DeclareMathOperator{\E}{\mathbb{E}}
\DeclareMathOperator{\Prb}{\mathbb{P}}
\DeclareMathOperator{\R}{\mathbb{R}}
\DeclareMathOperator{\N}{\mathbb{N}}
\DeclareMathOperator{\1}{\mathbbm{1}}
\newcommand{\ind}{\perp\!\!\!\!\perp}


\titleformat{\part}{\centering\normalfont\Large\bfseries}{\partname\hspace{5pt}\thepart\hspace{5pt}}{5pt}{--\ }

\titleformat{\part}{\centering\normalfont\Large\bfseries}{\partname\hspace{5pt}\thepart\hspace{5pt}}{5pt}{--\ }

\makeatletter
\def\@fnsymbol#1{\ensuremath{\ifcase#1\or \dagger\or \ddagger\or
   \mathsection\or \mathparagraph\or \|\or **\or \dagger\dagger
   \or \ddagger\ddagger \else\@ctrerr\fi}}
    \makeatother


\begin{document}

        \title{\scshape{Financial Econometrics 1 - M2 FTD \\ Empirical Applications}}
        \author{Luis Miguel Fonseca \\ Stéphane Eloundou Mvondo\\ Natalia Cárdenas Frías }
        \date{\today}
        \maketitle 

\tableofcontents
\newpage


\section*{Introduction} \addcontentsline{toc}{section}{\protect\numberline{}Introduction}
\textcolor{blue}{something, probably describe how all applications make sense one after the other and what is the research question we could have made ourselves when doing the applications, try to give a coherent look to the whole thing.}





\section{Series Dynamics}\label{sec:dynamics}




\section{Canonical VAR model application}\label{sec:var}




\section{Cointegration theory}\label{sec:cointegration}




\section{Impulse Response Analysis}\label{sec:irf}
    \subsection{Canonical IRF}\label{sec:canonical_irf}


    \subsection{Structural IRF}\label{sec:structural_irf}



\section{Introduce non-linearities}\label{sec:nonlinearities}
    \subsection{Markov-switching model}\label{sec:markov}

    
    \subsection{STR model}\label{sec:str}


    
\end{document}